\documentclass{article}
\usepackage[utf8]{inputenc}
\usepackage[portuguese]{babel}
\usepackage{hyperref}
\usepackage{listings}
\usepackage{color}
\usepackage{titlesec}

\setcounter{tocdepth}{5}

\definecolor{background}{rgb}{1,1,1}
\definecolor{lightgray}{RGB}{150,150,150}
\definecolor{darkmagenta}{RGB}{139,0,139}
\definecolor{purple}{rgb}{0.65, 0.12, 0.82}

% Javascript language definition.
\lstdefinelanguage{JavaScript}{
  keywords={break, case, catch, continue, debugger, default, delete, do, else, finally, for, function, if, in, instanceof, new, return, switch, this, throw, try, typeof, var, void, while, with},
  keywordstyle=\color{darkmagenta}\bfseries,
  ndkeywords={class, export, boolean, throw, implements, import, this, sky6StarChart, sky6RASCOMTele, sky6ObjectInformation, TextFile},
  ndkeywordstyle=\color{blue}\bfseries,
  identifierstyle=\color{black},
  sensitive=false,
  comment=[l]{//},
  morecomment=[s]{/*}{*/},
  commentstyle=\color{lightgray}\ttfamily,
  stringstyle=\color{red}\ttfamily,
  morestring=[b]',
  morestring=[b]"
}

\lstset{
   language=JavaScript,
   backgroundcolor=\color{background},
   extendedchars=true,
   basicstyle=\footnotesize\ttfamily,
   showstringspaces=false,
   showspaces=false,
   numbers=left,
   numberstyle=\footnotesize,
   numbersep=9pt,
   tabsize=2,
   breaklines=true,
   showtabs=false,
   captionpos=b
}

\title{Documentação do script de automatização do 7GHz}
\author{Edison Neto}

\begin{document}
\maketitle
\newpage
\tableofcontents
\newpage

\section{Introdução}

O script está escrito em Javascript, só é possível executá-lo usando o interpretador do SkyX.

\section{Funcionamento geral}

A rotina está dentro de um loop infinito que fica pegando o horário atual do computador e comparando com os horários pré-determinados para iniciar algum dos processo.

Antes do início do loop são definidos os horários (UT) para ligar, fazer o flip e desligar. Para o início, é verificada se a hora é exatamente a do horário de inicialização. Antes de fazer o Slew, é necessário usar a função FindHome. Como não há uma forma de saber se o telescópio já fez o home ou não, a função FindHome deve ser executada sempre na inicialização. O flip, como a inicialização, é realizado precisamente no horário determinado. O desligamento ocorre se o tracking estiver sendo realizado e se a hora atual for maior ou igual a hora de desligamento.

Se a conexão for perdida há a possibilidade dela ser recuperada e que o telescópio volte a sua rotina normal. Entretanto, o problema causado pela perda de conexão pode não ser resolvido, e há a possibilidade de que seja necessária um reconexão manual.

A documentação das funções usa o padrão \href{''http://usejsdoc.org/''}{JSDoc}


\section{Funções do script}

Algumas funções foram escritas com tratamento de erro, e com finalidade mais relacionada à automatização do 7GHz.

\subsection{Funções de utilidade geral}

\subsubsection{Logging}

Para fazer o logging existem três funções diferentes

\begin{lstlisting}
function WriteLog(level, text)
{
    var filename = setFileName();
    try {
        TextFile.openForAppend(filename);
        var formattedTime = getFormattedTime();
        var header = "[" + level + " - " + formattedTime  + "] ";
        TextFile.write(header + text + "\n");
        print(header + text);
        TextFile.close();
    } catch (texterr) {
        PrintAndOut("Erro ao editar o log. " + texterr.message + " (WriteLog)");
    }
}
\end{lstlisting}

\subsubsection{Escrevendo no debugger e no RunJavaScript}

Para escrever no debugger e na janela RunJavaScript a mesma mensagem em tempo de execução.

\begin{lstlisting}
function PrintAndOut(text)
{
    var level = "WARNING";
    var formattedTime = getFormattedTime();
    var header = "[" + level + " - " + formattedTime  + "] ";
    print(header + text);
    RunJavaScriptOutput.writeLine(header + text);
}
\end{lstlisting}

\subsubsection{Pegando a ascensão reta e a declinação do objeto}

Pega a ascensão reta e a declinação de algum objeto qualquer dentro do limite preestabelecido.

\begin{lstlisting}
function GetRADec(object)
{
    if (!Sky6IsConnected()) {
        WriteLogError("Erro de conexao (GetRaDec)");
        return false;
    }

    try {
        sky6StarChart.Find(object);
    } catch (finderr) {
        WriteLogError("Erro durante o find. " + finderr.message + " (GetRaDec)");
        return false;
    }

    sky6ObjectInformation.Property(54);
    var targetRA = sky6ObjectInformation.ObjInfoPropOut;
    sky6ObjectInformation.Property(55);
    var targetDec = sky6ObjectInformation.ObjInfoPropOut;

    return {"ra": targetRA, "dec": targetDec};
}
\end{lstlisting}

\subsection{Funções de controle}

\subsubsection{Conexão}

Inicia a conexão entre o SkyX e a montagem e cria o arquivo de log do dia. Essa função deve ser executada quando o SkyX não está conectado e for exatamente 11:00(UT), ou o horário escolhido para o início da execução. Esse processo é muito rápido comparado com outras operações de controle, demorando não mais de 1 segundo.

\begin{lstlisting}
function Connect_c()
{
    var time = getTimeNow();
    var formattedTime = getFormattedTime();

    ConnectTelescope();

    var filename = setFileName();
    TextFile.createNew(filename);
    TextFile.write(String(time.day) + "/" + String(time.month) + "/" + String(time.year) + "\n");
    var header = "[INFO - " + formattedTime + "] ";
    TextFile.write(header + "Conectado\n");
    print(header + "Conectado\n");
    TextFile.close();
}
\end{lstlisting}

\subsubsection{Inicialização}

Inicia o rastreamento do sol. Essa função dever ser executada quando o SkyX está conectado e for exatamente 11:00(UT). Pelo fato do processo de conexão ser muito rápido, não há necessidade de iniciar 1 ou 2 minuto(s) depois da conexão.

\begin{lstlisting}
function Initialize_c()
{
    if (sky6RASCOMTele.IsParked != 0) {
        sky6RASCOMTele.Unpark();
    }
    sky6RASCOMTele.FindHome();
    var props = GetRADec("Sun");

    WriteLogInfo("Iniciou o slew (Initialize_c)");
    SlewTelescopeToRaDec(props.ra, props.dec, "Sun");

    WriteLogInfo("Iniciou o rastreamento (Initialize_c)");
}

\end{lstlisting}

\subsubsection{Flip}

Faz o flip, basicamente reiniciando o rastreamento. A única diferença de código entre a função Initialize\_c, é a inutilidade da função FindHome, visto que sua execução somente é necessária uma única vez por dia (desconsiderando problemas de conexão).

\begin{lstlisting}
function Flip_c()
{
    var props = GetRADec("Sun");
    WriteLogInfo("Iniciou o slew (Flip_c)");
    SlewTelescopeToRaDec(props.ra, props.dec, "Sun");

    WriteLogInfo("Completou o flip (Flip_c)");
}
\end{lstlisting}

\subsubsection{Desligamento}

Desliga o rastreamento primeiro e depois vai para a posição de parking, desconectando logo em seguida. É executada quando o SkyX está conectado e já passou das 20:00(UT).

\begin{lstlisting}
function TurnOff_c()
{
    SetTelescopeTracking(0, 1, 0, 0);
    WriteLogInfo("Desligou o rastreamento (TurnOff_c)");

    ParkTelescope();
    WriteLogInfo("Desconectado (TurnOff_c)");
}
\end{lstlisting}

\subsubsection{Reconexão}

Reconecta o telescópio e reinicia o rastreamento. Também executa a função FindHome, já que se o script for (re)iniciado depois das 11:00, este processo pode não ter sido realizado. É executada quando o SkyX não está conectado e a hora atual está entre o horário de execução.

\begin{lstlisting}
function Reconnect_c()
{
    WriteLogInfo("(Re)conectado (Reconnect_c)");
    ConnectTelescope();
    SetTelescopeTracking(0, 1, 0, 0);
    sky6RASCOMTele.FindHome();
    RestartTracking_c();
}
\end{lstlisting}

\subsubsection{Reinicialização do tracking}

Reinicia o rastreamento. É executada na função de reconexão e quando o SkyX está conectado, não está fazendo o tracking e a hora atual está entre o horário de execução.

\begin{lstlisting}
function RestartTracking_c()
{
    var props = GetRADec("Sun");

    WriteLogInfo("Iniciou o slew (RestartTracking_c)");
    SlewTelescopeToRaDec(props.ra, props.dec, "Sun");

    WriteLogInfo("Reiniciou o rastreamento (RestartTracking_c)");
}
\end{lstlisting}

\subsubsection{Calibração}

Aponta para o céu somando 20 graus na altitude da observação atual. A calibração é feita duas vezes ao dia, uma hora antes do flip e uma hora depois.

\begin{lstlisting}
function CalibrateTelescope_c()
{
    WriteLogInfo("Calibracao iniciada (CalibrateTelescope_c)")

    var delta = 20;
    var props = GetAzAlt();
    WriteLogInfo("Azimute atual: " + props.az + " | Altitude atual: " + props.alt);
    var newAz = props.az + delta;
    WriteLogInfo("Azimute futuro: " + newAz + " | Altitude futura: " + props.alt);

    SlewTelescopeToAzAlt(newAz, props.alt, "");
}
\end{lstlisting}

\section{Guia de estilo do código}

\subsection{Indentação}

O código usa espaços com 4 caracteres de largura.

\subsection{Posicionamento das chaves}

A forma correta é colocar a chave de abertura por último na linha, e colocar a chave de fechamento primeiro.

\begin{lstlisting}
if (something === true) {
  print(something);
}
\end{lstlisting}

Para funções coloque a chave embaixo da próxima linha.

\begin{lstlisting}
function myFunction()
{
  return 0;
}
\end{lstlisting}

\subsection{Nomeando funções}

As funções que usam alguma classe do SkyX são nomeadas usando UpperCamelCase, já as que não usam são nomeadas usando lowerCamelCase. As função principais de controle, são nomeadas usando UpperCamelCase e com o sufixo `\_c'.

\section{Biblioteca do SkyX}

\subsection{Principais classes}

\begin{itemize}
    \item sky6ObjectInformation - Informações dos objetos
    \item sky6StarChart - Acesso aos aspectos visuais do SkyX
    \item sky6RASCOMTele - Controle físico da montagem
    \item TextFile - Manipulação de arquivos
\end{itemize}

\subsection{Exemplos}

\subsubsection{sky6ObjectInformation}

Dá o acesso ao banco de dados do SkyX. Podendo pegar diversas informações sobre o objeto sendo observado.

\paragraph{Property()}\mbox{}\\

\textbf{Argumentos}:

\begin{itemize}
    \item \emph{number} number - Representa um certa informação do objeto.
\end{itemize}

Há um total de 189 informações separadas nessa função. A função em si não retorna nada, o valor fica armazenado na propriedade \emph{ObjInfoPropOut}, como escrito no exemplo.

\textbf{Exemplo:}

\begin{lstlisting}
sky6ObjectInformation.Property(55);
print(sky6ObjectInformation.ObjInfoPropOut);
\end{lstlisting}

As duas propriedades que são usadas no script, são as seguintes:

\begin{itemize}
    \item 54: Ascensão reta
    \item 55: Declinação
\end{itemize}

\subsubsection{sky6StarChart}

Controle da parte visual do SkyX. Basicamente com essa classe é possível fazer o que se faria clicando no TheSkyX.

\paragraph{Find()}\mbox{}\\

\textbf{Argumentos}:

\begin{itemize}
    \item \emph{string} objectName - O nome do objeto a ser procurado.
\end{itemize}

Procura pelo objeto dado.

\textbf{Exemplo}:

\begin{lstlisting}
sky6StarChart.Find("Sun");
\end{lstlisting}

Exemplo usando o sky6ObjectInformation.Property():

\begin{lstlisting}
// Procura pelo sol.
sky6StarChart.Find("Sun");
// Prepara no ObjInfoPropOut o valor da declinacao.
sky6ObjectInformation.Property(55);

print(sky6ObjectInfomation.ObjInfoPropOut + "\n")
\end{lstlisting}

\subsubsection{sky6RASCOMTele}

Dá o controle físico da montagem. Para coisas como o slew ou parking.

\paragraph{Connect(void)}\mbox{}\\

Inicia a comunicação entre a montagem e o SkyX.

\paragraph{Disconnect(void)}\mbox{}\\

Termina a conexão entre a montagem e o SkyX.

\paragraph{Abort(void)}\mbox{}\\

Aborta a operação que estiver sendo realizada.

\paragraph{SlewToRaDec()}\mbox{}\\

\textbf{Argumentos}:

\begin{itemize}
    \item \emph{number} TargetRa - A ascensão reta;
    \item \emph{number} TargetDec - A declinação;
    \item \emph{string} targetObject - O nome do objeto.
\end{itemize}


Aponta o telescópio para a coordenada dada.

\begin{lstlisting}
sky6RASCOMTele.Connect();

var targetObject = "Sun";
sky6StarChart.Find(targetObject);

sky6ObjectInformation.Property(54);
var targetRA = sky6ObjectInformation.ObjInfoPropOut;

sky6ObjectInformation.Property(55);
var targetDec = sky6ObjectInformation.ObjInfoPropOut;
sky6RASCOMTele.SlewToRaDec(targetRa, TargetDec, targetObject);
\end{lstlisting}

\paragraph{GetRaDec(void)}\mbox{}\\

Pega a declinação e a ascensão reta atual, e prepara os valores nas variáveis dRa e dDec.

Exemplo:

\begin{lstlisting}
sky6RASCOMTele.Connect();
sky6RASCOMTele.GetRaDec();

print(String(sky6RASCOMTele.dRa) + " | ");
print(String(sky6RASCOMTele.dDec));
\end{lstlisting}

\paragraph{Park(void)}\mbox{}\\


Faz o slew para a posição de parking, e finaliza a conexão com o TheSky6.

\paragraph{ParkAndDoNotDisconnect(void)}\mbox{}\\

Tem quase o mesmo funcionamento que a função `Park'. A diferença é que essa função não finaliza a conexão entre o telescópio e o TheSky6.
Para fazer outro Slew depois de usar esta função é necessário utilizar a função `Unpark' antes.

\paragraph{Unpark(void)}\mbox{}\\

Tira o telescópio do estado de parked.

\paragraph{IsConnected}\mbox{}\\

Tem o valor zero se o telescópio não estiver conectado.

\paragraph{IsParked}\mbox{}\\

Tem o valor zero se o telescópio não estiver na posição de parking.

\paragraph{IsTracking}\mbox{}\\

Tem o valor zero se o telescópio não estiver fazendo o tracking.

\paragraph{dRa}\mbox{}\\

A ascensão reta atual.

\paragraph{dDec}\mbox{}\\

A declinação atual.

\subsubsection{TextFile}

Classe usada para manipulação básica de arquivos.

\paragraph{createNew()}\mbox{}\\

Cria um arquivo txt. Os arquivos são obrigatoriamente na pasta \emph{/Meus Documentos/Software Bisque/TheSkyX Professional Edition/ScriptFiles}, o local dos não pode ser modificado.O nome do arquivo só pode conter letras e números.

\textbf{Argumentos}:

\begin{itemize}
    \item \emph{string} filename - O nome do arquivo (sem a extensão).
\end{itemize}

\paragraph{write()}\mbox{}\\

Escreve uma string no arquivo.

\textbf{Argumentos}:

\begin{itemize}
    \item \emph{string} text - String a ser escrita no arquivo.
\end{itemize}

\paragraph{openForAppend()}\mbox{}\\

Abre o arquivo de forma a anexar novos conteúdos depois do que já está escrito. Caso o arquivo não tiver sido criado e essa função for usada, o arquivo será criado.

\textbf{Argumentos}:

\begin{itemize}
    \item \emph{string} filename - O nome do arquivo
\end{itemize}

\paragraph{close(void)}\mbox{}\\

Fecha o arquivo salvando as modificações feitas.

\subsubsection{Não relacionadas com as classes}

\paragraph{String()}\mbox{}\\

\textbf{Argumentos}:

\begin{itemize}
    \item \emph{number} int - Uma variável numérica qualquer.
\end{itemize}

Transforma um número em uma string. Essa função funciona como o método toString() do javascript(que curiosamente não funciona no SkyX).

\textbf{Exemplo}:

\begin{lstlisting}
var int = 2;
print(String(int));
\end{lstlisting}

\paragraph{print()}\mbox{}\\

\textbf{Argumentos}:

\begin{itemize}
    \item \emph{(string, number)} text - Uma variável qualquer.
\end{itemize}

Essa função escreve nos logs do debugger. Ela escreve enquanto o programa roda, diferentemente da variável Out.

\paragraph{A variável Out}\mbox{}\\

Essa variável armazena tudo que será escrito na tela do "Run Java Script" no SkyX. Ela só é escrita quando o script acaba de rodar. É possível escrever nessa tela em tempo de execução com a classe RunJavaScriptOutput.

\section{O código}

\subsection{Versão 1.5}

\begin{lstlisting}
/*
 * Version: 1.5
 */

/**
 * Confirma se o SkyX tem conexao com a montagem.
 *
 * @returns {boolean} false se nao estiver conectado.
 *                    true se estiver conectado.
 */
function Sky6IsConnected()
{
    if (sky6RASCOMTele.IsConnected == 0) {
        return false;
    }
    return true;
}

/**
 * Estabiliza a conexao com a montagem.
 *
 * @returns {boolean} false caso algum erro aconteca.
 */
function ConnectTelescope()
{
    try {
        sky6RASCOMTele.Connect();
    } catch (connerr) {
        WriteLogError("Erro de conexao com a montagem. " + connerr.message + " (ConnectTelescope)");
        return false;
    }

    return true;
}

/**
 * Liga/Desliga o tracking.
 *
 * @param {number} IOn - O numero que desliga ou liga o tracking.
 *                              0 - desliga
 *                              1 - liga
 *
 * @param {number} IIgnoreRates - O numero que especifica se e para o 
 *                                telescopio usar a taxa de tracking atual.
 *                              0 - Ignora os valores de dRaRate e dDecRate
 *                              1 - Usa os valores de dRaRate e dDecRate
 *
 * @param {number} dRaRate - Especifica a ascensao reta a ser usada.
 *                           So e utilizada se IIgnoreRates for igual a 1.
 *
 * @param {number} dDecRate - Especifica a declinacao a ser usada.
 *                           So e utilizada se IIgnoreRates for igual a 1.
 *
 * @returns {boolean} false se a montagem nao estiver conectada.
 */
function SetTelescopeTracking(IOn, IIgnoreRates, dRaRate, dDecRate)
{
    if (!Sky6IsConnected()) {
        return false;
    }

    sky6RASCOMTele.SetTracking(IOn, IIgnoreRates, dRaRate, dDecRate);
    return true;
}

/**
 * Faz o slew para um determinado objeto dados sua ascensao reta e declinacao.
 *
 * @param {number} dRa - ascensao reta.
 * @param {number} dDec - declinacao.
 * @param {string} targetObject - Objeto em questao.
 *
 * @returns {boolean} true se tudo tiver ocorrido corretamente.
 */
function SlewTelescopeToRaDec(dRa, dDec, targetObject)
{
    if (!Sky6IsConnected()) {
        WriteLogError("Telescopio nao conectado (SlewTelescopeToRaDec)");
        return false;
    }
  
    try {
        sky6RASCOMTele.SlewToRaDec(dRa, dDec, targetObject);
        return true;
    } catch (slewerr) {
        WriteLogError("Falha durante o slew. " + slewerr.message + " (SlewTelescopeToRaDec)");
        return false;
    }
}

/**
 * Faz o slew para um determinado objeto dados azimute e altitude.
 *
 * @param {number} az - Azimute.
 * @param {number} alt - Altitude.
 * @param {string} targetObject - Objeto em questao.
 *
 * @returns {boolean} true se tudo tiver ocorrido corretamente.
 */
function SlewTelescopeToAzAlt(az, alt, targetObject)
{
    if (!Sky6IsConnected()) {
        WriteLogError("Telescopio nao conectado. (SlewTelescopeToAzAlt)");
        return false;
    }

    try {
        sky6RASCOMTele.SlewToAzAlt(az, alt, targetObject);
        return true;
    } catch (slewerr) {
        WriteLogError("Falha durante o slew. " + slewerr.message + " (SlewTelescopeToAzAlt)");
        return false;
    }
}

/**
 * Leva o telescopio para a posicao de parking.
 *
 * @returns {boolean} true se tudo tiver ocorrido corretamente.
 */
function ParkTelescope()
{
    if (!Sky6IsConnected()) {
        return false;
    }

    if (sky6RASCOMTele.IsParked != 0) {
        sky6RASCOMTele.Park();
        WriteLogInfo("Parking finalizado (ParkTelescope)");
        return true;
    }
}

/**
 * Procura pelo objeto dado e pega a ascensao reta e a declinacao dele
 * no momento.
 *
 * @param {string} object - Nome do objeto a ser encontrado.
 *
 * @returns {object} Um objeto com a ascensao reta e a declinacao.
 */
function GetRADec(object)
{
    if (!Sky6IsConnected()) {
        WriteLogError("Erro de conexao (GetRaDec)");
        return false;
    }

    try {
        sky6StarChart.Find(object);
    } catch (finderr) {
        WriteLogError("Erro durante o find. " + finderr.message + " (GetRaDec)");
        return false;
    }

    sky6ObjectInformation.Property(54);
    var targetRA = sky6ObjectInformation.ObjInfoPropOut;
    sky6ObjectInformation.Property(55);
    var targetDec = sky6ObjectInformation.ObjInfoPropOut;

    return {"ra": targetRA, "dec": targetDec};
}

/**
 * Pega o azimute e a altitude do objeto sendo observado no momento.
 *
 * @returns {object} Um objeto com o azimute e a altitude.
 */
function GetAzAlt()
{
    if (!Sky6IsConnected()) {
        WriteLogError("Erro de conexao (GetAzAlt)");
        return false;
    }

    sky6RASCOMTele.GetAzAlt();
    var az = sky6RASCOMTele.dAz;
    var alt = sky6RASCOMTele.dAlt;

    return {"az": az, "alt": alt};
}

/**
 * Pega a data e o horario do momento que a funcao e chamada.
 *
 * @returns {object} Um objeto com os dados.
 */
function getTimeNow()
{
    var time = new Date();
    var day = time.getDate();
    var month = time.getMonth();
    var year = time.getFullYear();
    var hour = time.getHours();
    var minutes = time.getMinutes();
    var seconds = time.getSeconds();

    return {
        "day": day,
        "month": month,
        "year": year,
        "hour": hour,
        "minutes": minutes,
        "seconds": seconds
    };
}

/**
 * Cria o nome do arquivo para o dia atual.
 *
 * @returns {string} O nome do arquivo do dia atual.
 */
function setFileName()
{
    var time = getTimeNow();

    if (time.day < 10) {
        var day = "0" + String(time.day);
    } else {
        var day = String(time.day);
    }

    if (time.month < 10) {
        var month = "0" + String(time.month);
    } else {
        var month = String(time.month);
    }

    var year = String(time.year);

    var filename = year + month + day;

    return filename;
}

/**
 * Pega o horario atual do computador formatado.
 *
 * @returns {string} O horario no formato %H:%M:%S.
 */
function getFormattedTime()
{
    var time = getTimeNow();
    var formattedTime = String(time.hour) + ":" + String(time.minutes) + ":" + String(time.seconds);
    return formattedTime;
}

/**
 * Escreve no log uma mensagem, junto com o horario do momento.
 * Tambem define o tipo(nivel) de mensagem(Info, Warning ou Error).
 * Formato da mensagem: [LEVEL - 00:00:00] Text
 *
 * @param {string} text -  A mensagem a ser escrita.
 * @param {string} level - Nivel da mensage.
 */
function WriteLog(level, text)
{
    var filename = setFileName();
    try {
        TextFile.openForAppend(filename);
        var formattedTime = getFormattedTime();
        var header = "[" + level + " - " + formattedTime  + "] ";
        TextFile.write(header + text + "\n");
        print(header + text);
        TextFile.close();
    } catch (texterr) {
        PrintAndOut("Erro ao editar o log. " + texterr.message + " (WriteLog)");
    }
}

/**
 * Escreve a mensage de warning no log.
 *
 * @param {string} text -  A mensagem a ser escrita.
 */
function WriteLogWarning(text)
{
    WriteLog("WARNING", text);
}

/**
 * Escreve a mensagem de erro no log.
 *
 * @param {string} text - A mensagem a ser escrita.
 */
function WriteLogError(text)
{
    WriteLog("ERROR", text);
}

/**
 * Escreve a mensagem de informacao no log.
 *
 * @param {string} text - A mensagem a ser escrita.
 */
function WriteLogInfo(text)
{
    WriteLog("INFO", text);
}

/**
 * Escreve no debugger e na janela Run Java Script.
 * Deve ser usado somente quando o log estiver inacessivel.
 *
 * @param {string} text - O conteudo a ser escrito.
 */
function PrintAndOut(text)
{
    var level = "WARNING";
    var formattedTime = getFormattedTime();
    var header = "[" + level + " - " + formattedTime  + "] ";
    print(header + text);
    RunJavaScriptOutput.writeLine(header + text);
}

/**
 * Conecta o telescopio e cria o arquivo de log do dia.
 */
function Connect_c()
{
    var time = getTimeNow();
    var formattedTime = getFormattedTime();

    ConnectTelescope();

    var filename = setFileName();
    TextFile.createNew(filename);
    TextFile.write(String(time.day) + "/" + String(time.month) + "/" + String(time.year) + "\n");
    var header = "[INFO - " + formattedTime + "] ";
    TextFile.write(header + "Conectado\n");
    print(header + "Conectado\n");
    TextFile.close();
}

/**
 * Processo de inicializacao.
 */
function Initialize_c()
{
    if (sky6RASCOMTele.IsParked != 0) {
        sky6RASCOMTele.Unpark();
    }
    sky6RASCOMTele.FindHome();
    var props = GetRADec("Sun");

    WriteLogInfo("Iniciou o slew (Initialize_c)");
    SlewTelescopeToRaDec(props.ra, props.dec, "Sun");

    WriteLogInfo("Iniciou o rastreamento (Initialize_c)");
}

/**
 * Processo do Flip.
 */
function Flip_c()
{
    var props = GetRADec("Sun");
    WriteLogInfo("Iniciou o slew (Flip_c)");
    SlewTelescopeToRaDec(props.ra, props.dec, "Sun");

    WriteLogInfo("Completou o flip (Flip_c)");
}

/**
 * Desliga o tracking e faz o parking.
 */
function TurnOff_c()
{
    SetTelescopeTracking(0, 1, 0, 0);
    WriteLogInfo("Desligou o rastreamento (TurnOff_c)");

    ParkTelescope();
    WriteLogInfo("Desconectado (TurnOff_c)");
}

/**
 * Reconecta o telescopio e reinicia o tracking.
 */
function Reconnect_c()
{
    WriteLogInfo("(Re)conectado (Reconnect_c)");
    ConnectTelescope();
    SetTelescopeTracking(0, 1, 0, 0);
    sky6RASCOMTele.FindHome();
    RestartTracking_c();
}

/**
 * Reinicia o rastreamento.
 */
function RestartTracking_c()
{
    var props = GetRADec("Sun");

    WriteLogInfo("Iniciou o slew (RestartTracking_c)");
    SlewTelescopeToRaDec(props.ra, props.dec, "Sun");

    WriteLogInfo("Reiniciou o rastreamento (RestartTracking_c)");
}

/**
 * Aponta para o ceu baseando-se no azimute.
 */
function CalibrateTelescope_c()
{
    WriteLogInfo("Calibracao iniciada (CalibrateTelescope_c)")

    var delta = 20;
    var props = GetAzAlt();
    WriteLogInfo("Azimute atual: " + props.az + " | Altitude atual: " + props.alt);
    var newAz = props.az + delta;
    WriteLogInfo("Azimute futuro: " + newAz + " | Altitude futura: " + props.alt);

    SlewTelescopeToAzAlt(newAz, props.alt, "");
}

/*
 * Configura os horario para inicializar, fazer o flip e desligar.
 */
var work_time = {
    start_hour: 11,
    start_minutes: 00,
    start_seconds: 30,
    flip_hour: 16,
    flip_minutes: 00,
    turn_off_hour: 20,
    turn_off_minutes: 00,
    first_calibration_hour: 15,
    first_calibration_minutes: 00,
    first_calibration_seconds: 00,
    second_calibration_hour: 17,
    second_calibration_minutes: 00,
    second_calibration_seconds: 00,
    finish_first_calibration_hour: 15,
    finish_first_calibration_minutes: 00,
    finish_first_calibration_seconds: 30,
    finish_second_calibration_hour: 17,
    finish_second_calibration_minutes: 00,
    finish_second_calibration_seconds: 30,
};

/**
 * Verifica se e a hora da primeira calibracao.
 *
 * @param {object} time - Horario atual.
 * @returns {boolean}
 */
function timeToFirstCalibration(time)
{
    return time.hour == work_time.first_calibration_hour &&
                time.minutes == work_time.first_calibration_minutes &&
                time.seconds == work_time.first_calibration_seconds;
}

/**
 * Verifica se e a hora de voltar para o sol.
 * 
 * @param {object} time - Horario atual.
 * @return {boolean}
 */
function timeToFinishFirstCalibration(time)
{
    return time.hour == work_time.finish_first_calibration_hour &&
                time.minutes == work_time.finish_first_calibration_minutes &&
                time.seconds == work_time.finish_first_calibration_seconds;
}

/**
 * Verifica se e a hora da segunda calibracao.
 *
 * @param {object} time - Horario atual.
 * @returns {boolean}
 */
function timeToSecondCalibration(time)
{
    return time.hour == work_time.second_calibration_hour &&
                time.minutes == work_time.second_calibration_minutes &&
                time.seconds == work_time.second_calibration_seconds;
}

/**
 * Verifica se e a hora de voltar para o sol.
 * 
 * @param {object} time - Horario atual.
 * @return {boolean}
 */
function timeToFinishSecondCalibration(time)
{
    return time.hour == work_time.finish_second_calibration_hour &&
                time.minutes == work_time.finish_second_calibration_minutes &&
                time.seconds == work_time.finish_second_calibration_seconds;
}

/**
 * Verifica se e a hora de inicializar.
 *
 * @param {object} time - Horario atual.
 * @returns {boolean}
 */
function timeToInitialize(time)
{
    return time.hour == work_time.start_hour &&
                time.minutes == work_time.start_minutes &&
                time.seconds == work_time.start_seconds;
}

/**
 * Verifica se e a hora do flip.
 *
 * @param {object} time - Horario atual.
 * @returns {boolean}
 */
function timeToFlip(time)
{
    return time.hour == work_time.flip_hour &&
                time.minutes == work_time.flip_minutes;
}

/**
 * Verifica se e(ou ja passou) (d)a hora de desligar o tracking e se ele
 * esta ocorrendo.
 *
 * @param {object} time - Horario atual.
 * @returns {boolean}
 */
function timeToTurnOff(time)
{
    return time.hour >= work_time.turn_off_hour &&
                sky6RASCOMTele.IsTracking != 0;
}

/**
 * Verifica se e a hora de iniciar a conexao.
 *
 * @param {object} time - Horario atual.
 * @returns {boolean}
 */
function timeToConnect(time)
{
    return time.hour == work_time.start_hour &&
                time.minutes == work_time.start_minutes;
}

/**
 * Verifica se o telescopio esta no horario de operacao.
 * Procura prever um eventual problema de simples desconexao do SkyX.
 *
 * @param {object} time - Horario atual.
 * @returns {boolean}
 */
function inOperatingTime(time)
{
    return time.hour >= work_time.start_hour &&
                time.hour < work_time.turn_off_hour;
}

while (true)
{
    var time = getTimeNow();

    if (Sky6IsConnected()) {
        if (timeToInitialize(time)) {
            Initialize_c();
        }
        else if (timeToFirstCalibration(time)) {
            CalibrateTelescope_c();
        }
        else if (timeToFinishFirstCalibration(time)) {
            RestartTracking_c();
        }
        else if (timeToFlip(time)) {
            Flip_c();
        }
        else if (timeToSecondCalibration(time)) {
            CalibrateTelescope_c();
        }
        else if (timeToFinishSecondCalibration(time)) {
            RestartTracking_c();
        }
        else if (timeToTurnOff(time)) {
            TurnOff_c();
        }
    }
    else if (timeToConnect(time)) {
        Connect_c();
    }
    else if (inOperatingTime(time)) {
        Reconnect_c();
    }
}
\end{lstlisting}

\subsection{Versão mais recente}

\href{''https://github.com/3ldr0n/Automatiza-o-7ghz/''}{https://github.com/3ldr0n/Automatiza-o-7ghz/}

\end{document}
